\documentclass{article}

%% 开头的设置可以确保中文随便打,并且能支持用LuaLatex编译。lyluatex只能通过LuaLatex来编译
\usepackage[UTF8]{ctex}

%% 使用lyluatex宏包,用以编译lilypond语句
\usepackage{lyluatex}

% geometry宏包可以用于灵活调整纸张大小
\usepackage{geometry}
\geometry{a4paper, scale=0.65}

%% 因为怕太密集,调整一下行距
\linespread{2.0}



%% 正文
\begin{document}
\title{使用lyluatex实现谱文混排}
\author{lilypond 手残粉}

\maketitle


搭建好lilypond和latex的设置之后,就可以实现各种方式的打谱。例如:

只需要输入 \textbackslash lilypond\{c' d' e'\},就能实现行中打谱 \lilypond{c' d' e'} 正如你所看到的那样。

大括号里的语法和lilypond里的语法一致。例如,输入\textbackslash lilypond\{  \textbackslash clef bass  \textbackslash omit Staff.TimeSignature
c' d' e'  \} 就可以看到变成\underline{低音谱号}以及\underline{去掉拍号}后\lilypond{\clef bass \omit Staff.TimeSignature c' d' e'}的打谱效果。

也可以使用\textbackslash begin\{lilypond\} 和 \textbackslash end\{lilypond\} 直接进入lilypond环境,像编辑数学公式一样编辑乐谱。

%% 空了一行
\vspace*{0.5\baselineskip}


%% lilypond 环境

\begin{lilypond}
    music = \relative {
        c d e
    }
    \score {
        \new ChoirStaff \with {
            instrumentName = "2 Fl."
        } <<
        \new Staff \transpose c c' \music
        \new Staff {
                \clef bass
                \music
            }
        >>
    }
\end{lilypond}

这样,既可以利用Latex的编排优势,又可以利用lilypond的打谱优势,lilypond笨拙的文字排版终于可以说拜拜了。

\end{document}